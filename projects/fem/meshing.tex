\documentclass[11pt]{article}
\usepackage{mathptmx}
\usepackage[small,compact]{titlesec} 
\usepackage{geometry}                % See geometry.pdf to learn the layout options. There are lots.
\usepackage{url}
\usepackage{hyperref}
\usepackage{breakurl}
\geometry{letterpaper}                   % ... or a4paper or a5paper or ... 
%\geometry{landscape}                % Activate for for rotated page geometry
\usepackage[parfill]{parskip}    % Activate to begin paragraphs with an empty line rather than an indent
\usepackage{graphicx}
\usepackage{amssymb, amsmath}
\usepackage{epstopdf}

\DeclareGraphicsRule{.tif}{png}{.png}{`convert #1 `dirname #1`/`basename #1 .tif`.png}

\title{Observations on Meshes}
\author{Miles Macklin}
%\date{}                                           % Activate to display a given date or no date

\begin{document}
\maketitle

\section{Motivation}

Manifold meshes are important for many purposes including physical simulation. 

\includegraphics[width=100mm]{Figures.pdf} 

\subsection{Triangle Meshes}

How to classify a vertex based on the number of incident faces and edges.

Interior Vertex: $numEdges = numFaces$

Boundary Vertex: $numEdges = numFaces+1$ or $numEdgesOpen = 2$
s
Otherwise vertex is singular.

\subsection{Tetrahedral Meshes}

There are analogous formulas for tetrahedral meshes, but I can't find them.

Interior Vertex: $numFaces = 2*numTets$

Boundary Vertex: ????

\section {References}

\begin{itemize}
\item{[Gueziec et. al] - Converting Sets of Polygons to Manifold Surfaces by Cutting and Stitching - \url{(http://mesh.brown.edu/taubin/pdfs/gueziec-etal-vis98.pdf}}
\end{itemize}
\end{document}  





























