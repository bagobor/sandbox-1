\documentclass[11pt]{article}
\usepackage{mathptmx}
\usepackage[small,compact]{titlesec} 
\usepackage{geometry}                % See geometry.pdf to learn the layout options. There are lots.
\usepackage{url}
\usepackage{hyperref}
\usepackage{breakurl}
\geometry{letterpaper}                   % ... or a4paper or a5paper or ... 
%\geometry{landscape}                % Activate for for rotated page geometry
\usepackage[parfill]{parskip}    % Activate to begin paragraphs with an empty line rather than an indent
\usepackage{graphicx}
\usepackage{amssymb, amsmath}
\usepackage{epstopdf}

\DeclareGraphicsRule{.tif}{png}{.png}{`convert #1 `dirname #1`/`basename #1 .tif`.png}

\title{Implicit Springs}
\author{Miles Macklin}
%\date{}                                           % Activate to display a given date or no date

\begin{document}
\maketitle

\section{Implicit Integration}
The backward Euler integration method described in [1][2] requires the Jacobian of spring forces with respect to a particle's position and velocity. This document shows how to derive these Jacobians for use in a semi-implicit integrator.

\subsection{Vector Calculus Basics}

\renewcommand{\v}[1]{\ensuremath{\mathbf{#1}}} % for vectors
\newcommand{\uv}[1]{\ensuremath{\mathbf{\hat{#1}}}} % for unit vectors
\newcommand\ddx[1]{\frac{\partial#1}{\partial \v{x} }} 
\newcommand\dd[2]{\frac{\partial#1}{\partial #2}} 

In order to calculate the force Jacobians we will need to know how to calculate the derivatives of some basic quantities with respect to a vector.

In general the derivative of a scalar valued function with respect to a vector is defined as:

\[\ddx{f} = \begin{bmatrix} \dd{f}{x_i} & \dd{f}{x_j} & \dd{f}{x_k} \end{bmatrix}\]

And for a vector valued function with respect to a vector:

\[\ddx{\v{f}} = \begin{bmatrix} \dd{f_i}{x_i} & \dd{f_i}{x_j} & \dd{f_i}{x_k} \\ \dd{f_j}{x_i} & \dd{f_j}{x_j} & \dd{f_j}{x_k} \\ \dd{f_k}{x_i} & \dd{f_k}{x_j} & \dd{f_k}{x_k} \end{bmatrix}\]

From these definitions we can work out the derivative of some basic geometric quantities. First, the derivative of a dot product of two vectors with respect to one vector:

\[\ddx{\v{x}^T \cdot \v{y}} = \v{y}^T \]

We will explicitly keep track of whether vectors are row vectors or column vectors as it will be important when taking derivatives of spring forces.

The derivative of a vector magnitude with respect to the vector, gives the normalized vector transposed: 

\[\ddx{|\v{x}|} = \left(\frac{\v{x}}{|\v{x}|}\right)^T = \uv{x}^T \]


The derivative of a normalized vector $\v{\hat{x}} = \frac{\v{x}}{|\v{x}|} $ can be obtained using the quotient rule:

\[\ddx{\uv{x}} = \frac{\v{I}|\v{x}| - \v{x}\cdot\uv{x}^T}{|\v{x}|^2}\]

Where $\v{I}$ is the $n$ x $n$ identity matrix where n is dimension of $x$, and the product of a column vector and a row vector $\uv{x}\cdot\uv{x}^T$ is the outer product which is an $n$ x $n$ matrix that can be constructed using standard matrix multiplication definition.

Dividing through by $|\v{x}|$ we have:

\[\ddx{\uv{x}} = \frac{\v{I} - \uv{x}\cdot\uv{x}^T}{\v{|x|}}\]

\subsection{Jacobian of Stretch Force}

Recall the equation for the stretch force on a particle $i$ due to an undamped Hookean spring:

\[\v{F_s} = -k_s(|\v{x}_{ij}| - r)\uv{x}_{ij}\]

Where $\v{x}_{ij} = \v{x}_i - \v{x}_j$ is the vector between the two connected particles positions and $r$ is the rest length.

The Jacobian of the stretch force with respect to particle $i$'s position is obtained by using the product rule for the two $\v{x}_i$ dependent terms in $\v{F_s}$:

\[\dd{\v{F_s}}{\v{x}_i} = -ks\left[(|\v{x}_{ij}| - r)\dd{\uv{x}_{ij}}{\v{x}_i} + \uv{x}_{ij}\dd{(|\v{x}_{ij}| - r)}{\v{x}_i}\right]\]

Using the previously derived formulas for the derivative of a vector magnitude and normalized vector we have:

\[\dd{\v{F_s}}{\v{x}_i} = -ks\left[(|\v{x}_{ij}| - r)\left(\frac{\v{I} - \uv{x}_{ij}\cdot \uv{x}_{ij}^T}{|\v{x}_{ij}|}\right) + \uv{x}_{ij}\cdot \uv{x}_{ij}^T\right]\]

Dividing the first two terms through by $|\v{x}_{ij}|$:

\[\dd{\v{F_s}}{\v{x}_i} = -ks\left[(1 - \frac{r}{|\v{x}_{ij}|})\left(\v{I} - \uv{x}_{ij}\cdot \uv{x}_{ij}^T\right) + \uv{x}_{ij}\cdot \uv{x}_{ij}^T\right]\]

Due to the symmetry in the definition of $\v{x}_{ij}$ we have:

\[\dd{\v{F_s}}{\v{x}_j}  = -\dd{\v{F_s}}{\v{x}_i}\]


\subsection{Jacobian of Damping Force}

The equation for the damping force on a particle $i$ due to a spring:

\[\v{F_d} = -k_d\cdot\uv{x}(\v{v}_{ij}\cdot \uv{x}_{ij})\]

Where $\v{v}_{ij} = \v{v}_i-\v{v}_j$ is the relative velocities of the two particles. This is the preferred formulation because it damps only velocities along the spring axis.

Taking the derivative with respect to $\v{v}_i$:

\[\dd{\v{F_d}}{\v{v}_i} = -k_d\cdot\uv{x}\cdot\uv{x}^T\]

As with stretching, the force on the opposite particle is simply negated:

\[\dd{\v{F_d}}{\v{v}_j} = -\dd{\v{F_d}}{\v{v}_i} \]

\section {References}

\begin{itemize}
\item{[Baraff Witkin] - Physically Based Modelling, SIGGRAPH course - \url{http://www.pixar.com/companyinfo/research/pbm2001/}}
\item{[Baraff Witkin] - Large Steps in Cloth Simulation - \url{http://run.usc.edu/cs599-s10/cloth/baraff-witkin98.pdf}}
\item{[N. Joubert] - An Introduction to Simulation (\url{http://njoubert.com/teaching/cs184_sp09/section/simulation.pdf})}
\item{[D Prichard] - Implementing Baraff and Witkin's Cloth Simulation - \url{http://davidpritchard.org/freecloth/docs/report.pdf}}
\item{[Choi] - Stable but Responsive Cloth - \url{http://graphics.snu.ac.kr/~kjchoi/publication/cloth.pdf}}
\item{Numerical Recipes, 3rd edition 2007 - ch17.5}
\end{itemize}
\end{document}  





























